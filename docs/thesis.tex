
% VYSOKOŠKOLSKÁ KVALIFIKAČNÍ PRÁCE
% autor: Michal Struna
% název: Umělá inteligence pro detekci exoplanet

\documentclass[a4paper,12pt]{article}

\usepackage{utils}
\usepackage{titletoc}


\expandafter\def\expandafter\UrlBreaks\expandafter{\UrlBreaks%  save the current one
  \do\-}


% ÚDAJE O PRÁCI
\def\jmenoFakulty{Fakulta elektrotechniky a informatiky}
\def\jmenoAutora{Michal Struna}
\def\nazevPrace{Umělá inteligence pro detekci exoplanet}
\def\typPrace{Diplomová práce}
\def\rok{2021}
\def\prefixZadani{img/zadani}	% cesta a začátek jména, bude doplněno číslo strany
\def\suffixZadani{.jpg}		% doplní se ke každému jménu souboru zadání
\def\datumOdevzdaniPrace{9.\,5.\,2021}

\long\def\textPodekovani{...}


\long\def\anotace{
...
}
\def\klicovaSlova{exoplanety, extrasolární planety, kepler, umělá inteligence, python}
\def\title{Artificial intelligence for exoplanet detection from transit data}
\long\def\annotation{
...
}
\def\keywords{exoplanet, extrasolar planets, kepler, artificial intelligence, python}

%%%%%%%%%%%%%%%%%%%%%%%%%%%%%%%%%%%%%%%%%%%%%%%%%%%%%%%%%%%%
% ZAČÁTEK DOKUMENTU
%%%%%%%%%%%%%%%%%%%%%%%%%%%%%%%%%%%%%%%%%%%%%%%%%%%%%%%%%%%%

\begin{document}

\deskpage
\mainpage
\assignment
\statement
\acknowledgment
\annotationcs	
\annotationen
\content
\imglist
\tablelist
\codelist
\formulalist
\shortlist

\begin{description}[font=\mdseries,leftmargin=6em,labelwidth=!,]
\item[ly]		Light year
\item[AU]		Astronomical unit
\end{description}

\clearpage\pagestyle{plain}\phantomsection\addcontentsline{toc}{section}{Úvod}
\section*{Úvod}
\label{uvod}

\section{Exoplanety}
\subsection{Důvody hledání}

\subsection{Statistické údaje o exoplanetách}

\section{Metody hledání exoplanet}

Exoplanety není téměř vůbec možné pozorovat přímo vizuálně, protože neemitují žádné světlo a~nachází se ve velkých vzdálenostech od Země. Lze ovšem pozorovat jejich působení na blízké hvězdy nebo jiné útvary, jež jsou pro nás viditelné. K~tomuto účelu se používá několik metod popsaných v této kapitole.

Zdaleka nejvýznamnější je tranzitní metoda, kterou byla objevena většina exoplanet. Velké množství planet bylo objeveno taktéž metodou radiálních vzdáleností. Pomocí ní byly objevovány planety především blízko Země.

\img[1]{Vlastnosti planet v závislosti na metodě objevení}{stats/count_by_method.pdf}

\dataplot

Málo hmotné planety byly objevovány jak tranzitní metodou, tak i~metodou radiálních rychlostí. Naproti tomu hmotné planety blízko své hvězdy byly objeveny především tranzitní metodou, kdežto hmotné planety daleko od své hvězdy metodou radiálních rychlostí. Nejvzdálenější planety od hvězdy byly pak objevovány metodou přímého zobrazení.

\img[1]{Závislost hmotnosti, periody oběhu a metody objevení planety}{stats/period_by_mass_by_method.pdf}

\dataplot

\clearpage
\subsection{Tranzitní metoda}

Někdy se může exoplaneta při obíhání dostat mezi svou hvězdu a~Zemi. Tento jev se pro pozorovatele na Zemi projeví jako mírný pokles jasu této hvězdy (obvykle v jednotkách procent). Pokud je hvězda teleskopem sledována dlouhodobě, je možné v~těchto změnách jasu hvězdy odhalit opakující se složku. To by mohlo indikovat přítomnost nějakého tělesa v~blízkosti této hvězdy.

Tyto změny však nemusí být vždy stejně velké a~na první pohled viditelné. V~systému hvězdy může být více planet, které hvězdu ovlivňují v~různých periodických intervalech, nebo jasnost hvězdy mohou ovlivňovat i~jiné jevy, než obíhající tělesa.

\img[1]{Tranzitní metoda}{img/transit.png}

\drawgimp

Tranzitní metoda vzbuzuje velký zájem především kvůli možnosti objevovat i~malé planety podobné Zemi~--~takové planety by mohly spíše splňovat podmínky pro život. Úspěšnost této metody však silně závisí na geometrii přechodu planety přes hvězdu a~poklesu jasu hvězdy.

\subsubsection{Měření jasu hvězdy}
\subsubsection{Výpočet údajů o~exoplanetě}

\clearpage
\subsection{Metoda radiálních rychlostí}

Stejně jako hvězda ovlivňuje obíhající planetu, tak i~planeta gravitačně ovlivňuje svou hvězdu a~obě tělesa obíhají kolem společného těžiště. Tento pohyb se může projevit jako opakované přibližování a~vzdálování hvězdy vůči pozorovateli na Zemi.~\cite{methods}

Pokud se zdroj elektromagnetického vlnění (hvězda) přibližuje vůči pozorovateli, vlnění má menší vlnovou délku a~jeví se více do modra, protože právě modrá barva má z~viditelného spektra nejmenší vlnovou délku. Obdobná situace nastává při vzdálování se zdroje vlnění od pozorovatele. Vlnová délka se zvětšuje a~barva jde do červena. Tomuto efektu se říká červený (resp. modrý) posuv.~\cite{methods}

\img[1]{Metoda radiálních vzdáleností}{img/radial_velocity.png}
\drawgimp

Dopplerův jev tvrdí, že podobný efekt lze zaznamenat nejen u elektromagnetického vlnění, ale i~u~vlnění prostředí (zvuku). Pokud se k~nám zdroj zvuku přibližuje, zvuk zpravidla vnímáme vyšším tónem, protože má menší vlnovou délku (vyšší frekvenci). Při vzdalování zdroje má zvuk větší vlnovou délku a je vnímán hlubším tónem. [TODO]

Opakující se změny ve vlnové délce jejího záření tak mohou být důsledkem obíhajícího tělesa kolem této hvězdy.~\cite{methods}

\subsubsection{Měření vlnové délky hvězdy}

TODO: Rozklad na složky.

\subsubsection{Výpočet radiální rychlosti hvězdy}

Vztah radiální rychlosti a~změny vlnové délky je dán následujícím vzorcem:

\formula{Změna vlnové délky záření na základě radiální rychlosti zdroje} {\Delta\lambda = \lambda_0 * \frac{v}{c}}{
\begin{tabular}{lll}
	$\Delta\lambda$ = změna vlnové délky & $\lambda_0$ = klidová vlnová délka & v = radiální rychlost \
\end{tabular}
}

Z~něj lze po vyjádření vypočítat radiální rychlost. V~tabulce níže je uvedena amplituda radiální rychlosti několika reálných hvězd.

% TODO
\tab{Radiální rychlosti hvězd v~závislosti na změnách vlnové délky záření}{
	\begin{tabular}{|l|l|l|l|}
		\hline
		\rowcolor{lightgray}
		Hvězda (složka) & $\lambda_0 $ [nm] & $\Delta\lambda$ [nm] & $\Delta\ v_{max}$ [$\frac{m}{s}$] \\
		\hline
		Slunce (1) & 0 & 0 & 0 \\
		\hline
		Slunce (2) & 0 & 0 & 0 \\
		\hline
		51 Pegasi b & 0 & 0 & 0 \\
		\hline
	\end{tabular}
}

\subsubsection{Výpočet hmotnosti exoplanety}

Ze vzorce níže lze odvodit hmotnost planety. Ke skutečné hmotnosti lze ovšem dojít pouze pokud známe sklon dráhy planety vůči pozorovateli $sin(i)$. To v~praxi příliš často neplatí, a~proto metoda radiálních rychlostí umožňuje výpočet pouze dolní hranice hmotnosti planety.

\formula[\cite{methods}]{Radiální rychlost}{\Delta v_{max} = \sqrt[3]{\frac{2\pi G}{T}} * \frac{M_p * sin(i)}{\sqrt[3]{(M_p + M_s)^2}} * \frac{1}{\sqrt{1 - e^2}}}{
\begin{tabular}{lll}
	$\Delta v_{max}$ = amplituda změny rychlosti & $M_s$ = hmotnost hvězdy & $M_p$ = hmotnost planety \\
	sin(i) = sklon dráhy vůči pozorovateli & e = excentricita dráhy & T = oběžná doba \\
\end{tabular}
}

Níže je uveden příklad výpočtu hmotností několika exoplanet. Pro jednoduchost předpokládejme, že se pozorovatel nachází přímo v~rovině oběžné dráhy planety.

{ % TODO
\catcode`\-=12
\tab{Příklady ovlivňování radiální rychlosti hvězd planetami}{
	\begin{tabular}{|l|l|l|l|l|l|l|l|}
		\hline	
		\rowcolor{lightgray}
		Těleso & Hvězda & $M_p$ [kg] & $M_s$ [kg] & sin(i) & e & T [r] & \textbf{$\Delta v_{max}$ [$\frac{m}{s}$]} \\	
		\hline	
		Země & \multirow{3}{*}{Slunce} & $5,97 * 10^{24}$ & \multirow{3}{*}{$2 * 10^{30}$} & \multirow{5}{*}{1} & 0,017 & 1 & \textbf{0,089} \\
		\cline{1-1}\cline{3-3}\cline{6-8}
		Jupiter & & $1,9 * 10^{27}$ & & & 0,048 & 11,86 & \textbf{12,4} \\
		\cline{1-1}\cline{3-3}\cline{6-8}
		\cline{1-1}\cline{3-3}\cline{6-8}
		Pluto & & $1,3 * 10^{22}$ & & & 0,247 & 247,41 & \textbf{0,00003} \\
		\cline{1-1}\cline{3-3}\cline{6-8}
		\cline{1-4}\cline{6-8}


		$\alpha$ Cen Bb & $\alpha$ Cen B & $6,75 * 10^{24}$ & $1,8 * 10^{30}$ & & 0 & 0,0089 & \textbf{0,51} \\	
		\cline{1-4}\cline{6-8}
		51 Pegasi b & 51 Pegasi & $0,88 * 10^{27}$ & $2,22 * 10^{30}$ & & 0,013 & 0,0116 & \textbf{55,9}  \\		
		\hline
	\end{tabular}
}

\clearpage
\subsection{Astrometrická metoda}

Astrometrická metoda využívá stejné vlastnosti vzájemného působení těles jako metoda radiálních rychlostí. Namísto zkoumání vlnové délky záření se však zaměřuje na polohu hvězdy. Hvězda, kolem níž obíhá dostatečně hmotné těleso, se bude v~důsledku působení tělesa nepatrně vychylovat ze své pozice~--~bude obíhat kolem těžiště soustavy.~\cite{methods}

\img[1]{Astrometrická metoda}{img/astrometry.png}

Pohyb hvězdy tak není přímočarý, ale vlnitý. Kolísání hvězdy je však pro pozorovatele na Zemi často pouze v~řádu stovek úhlových mikrovteřin až jednotek milivteřin.~\cite{methods} Z~tohoto důvodu byla astrometrickou metodou dosud objevena pouze jediná exoplaneta.~\cite{nasadata}

\img[1]{Kolísání hvězdy Gliese~876~s~planetou}{img/astrometry_trajectory.png}
\drawgimp

Dá se však očekávat, že se zlepšující se technikou bude tato metoda úspěšnější.

\tab{Ukázková tabulka}{
	\begin{tabular}{|l|l|c|c|c|c|c|c|}
		\hline	
		Data -0& Data 1& Data 2 & Data 3 & Data 4 & Data 5 \\
		\hline	
		Data 0& Data 1& Data 2 & Data 3 & Data 4 & Data 5 \\
		\hline	
		Data 0& Data 1& Data 2 & Data 3 & Data 4 & Data 5 \\
		\hline
	\end{tabular}
}

\lstinputlisting[caption={Spuštění programu}]{codes/usage.txt}

\subsection{Gravitační mikročočky}
\subsection{Přímé zobrazení}
\subsection{Pulsar timing}

\section{Satelity hledající exoplanety}
\subsection{Kepler}
\subsubsection{Mise K2}
\subsection{TESS}

\section{Umělá inteligence}

\section{Použité technologie}
\section{Návrh~a~vývoj aplikace}
\section{Rozvržení aplikace}
\section{Problémy řešené při implementaci}

\clearpage\pagestyle{plain}\phantomsection\addcontentsline{toc}{section}{Závěr}
\section*{Závěr}

\addcontentsline{toc}{section}{Použitá literatura}
\begin{thebibliography}{99}	% parametr určuje nejširší položku

\onlinesource{methods}{Metody objevování planet}{Astronomia}{23.~1.~2013}{~23.~12.~2019}{http://hvezdy.astro.cz/exoplanety/51-metody-objevovani-planet}

\onlinesource{nasadata}{NASA Exoplanet Archive}{NASA Exoplanet Science Insititute}{12.~8.~2019}{25.~12.~2019}{https://exoplanetarchive.ipac.caltech.edu/docs/program_interfaces.html}

\onlinesource[LOVIS, Christophe, FISCHER, Debra A]{radialvelocity}{Radial Velocity}{Yale Astronomy}{??.~?.~????}{25.~12.~2019}{http://exoplanets.astro.yale.edu/workshop/EPRV/Bibliography_files/Radial_Velocity.pdf}

\end{thebibliography}

Exoplanets data \url{http://exoplanets.org/detail/alpha_Cen_B_b}

Wavelength \url{http://spiff.rit.edu/classes/phys240/lectures/expand/expand.html}

\end{document}